
Having multiple assets, we have multiple threat sources. We are well aware that the application do lack a lot of security, but since this has not been the main focus of the course, it has not been our highest priority. \newline 
We have the following assets:
\begin{itemize}
    \item Web application
    \item Database
    \item CI/CD
    \item Image repository
    \item Container orchestration
    \item Secrets
\end{itemize}

\subsubsection{Threat sources}

Web application: Our current web application runs on HTTP, leaking data (especially when signing up) and making our application vulnerable to replay attacks. 
All dependencies for the application are Microsoft's own packages.
Messages are rendered in HTML, it is possible that this somehow allows for cross-site scripting (XSS).\\\\
Database: In the database the username and email is in clear-text, passwords are salted and hashed to defend against rainbow-table attacks.
The user ID is an integer, incremented on each sign-up, this is not best practice.
To connect to the database, we use the same user and password for both our staging and production environments.\\\\
CI/CD: GitHub actions has access to all of our secrets.
We use the following NPM packages for versioning our releases:
\begin{itemize}
    \item "semantic-release-github-actions-tags": "\textasciicircum1.0.3"
    \item "semantic-release/git": "\textasciicircum8.0.0"
\end{itemize}

Our images are stored publicly on DockerHub.
On our staging server, we don’t use tokens for logging into DockerHub, instead the credentials are stored in the environment, in clear text.
Our codebase is public on GitHub.\\\\
Other: We have a lot of open ports on our servers, possibly also for our logging endpoint.

\subsubsection{Risk scenarios}

\begin{center}
\begin{tabular}{ |p{3.5cm}|c|c|c|p{2.5cm}| } 
 \hline
 Scenario & Probability & Impact & Risk & Strategy / Action\\ [0.5ex] 
 \hline
 An adversary listens to the non-encrypted data sent on our web application. This is possible because we use HTTP and not HTTPS
 & 5 & 5 & 25 & Get an SSL-certificate and redirect all traffic to HTTPS\\
\hline
 An adversary finds an open port on one of our nodes and gains access to our system & 3 & 5 & 15 & Run Nmap, close unused ports\\
 \hline 
 An adversary constructs a message that escapes HTML and runs code when rendered (XSS) & 3 & 5 & 15 & Sanitize user input\\
 \hline
  An adversary finds our Docker images online & 3 & 1 & 3 & Make image repository private\\ 
 \hline
 An adversary hijacks the cookie of a user & 1 & 2 & 2 & Unavoidable\\ 
 \hline
\end{tabular}
\end{center}

\todo{brief description of how we harden the security of our system based on the analysis}