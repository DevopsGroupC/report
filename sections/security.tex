
\subsubsection {Risk identification}
We have the following assets: 
\begin{itemize}
    \item Web application
    \item Database
    \item CI/CD
    \item Image repository
    \item Container orchestration
    \item Secrets
\end{itemize}

\subsubsection{Threat sources}

\textit{Web application:} Runs on HTTP, making it vulnerable to data leaks and replay attacks. Messages are rendered in HTML, it is possible that this somehow allows for cross-site scripting (XSS).\\\\
\textit{Database:} Usernames and emails are stored in clear-text; passwords are salted and hashed, to defend against rainbow-table attacks.
User IDs increment sequentially, which is not best practice. To connect to the database, we use the same user and password for both our staging and production environments.\\\\
\textit{CI/CD:} GitHub actions has access to all secrets. Images are publicly stored on DockerHub. On our staging server, we don’t use tokens for logging into DockerHub, instead the credentials are stored in the environment, in clear text. Our codebase is public on GitHub.\\\\
% We use the following NPM packages for release versioning:
% \begin{itemize}
%     \item "semantic-release-github-actions-tags": "\textasciicircum1.0.3"
%     \item "semantic-release/git": "\textasciicircum8.0.0"
% \end{itemize}
\noindent\textit{Other:} We have multiple open ports on our servers, potentially also for the logging endpoint.\\

\noindent Additionally we are using multiple dependencies, which possesses a risk as well, as there could be vulnerabilities in connection to those. 

\subsubsection{Risk scenarios}

\begin{center}
\begin{tabular}{ |p{3.5cm}|c|c|c|p{2.5cm}| } 
 \hline
 Scenario & Probability & Impact & Risk & Strategy / Action\\ [0.5ex] 
 \hline
 An adversary listens to the non-encrypted data sent on our web application. This is possible because we use HTTP and not HTTPS
 & 5 & 5 & 25 & Get an SSL-certificate and redirect all traffic to HTTPS\\
\hline
 An adversary finds an open port on one of our nodes and gains access to our system & 3 & 5 & 15 & Run Nmap, close unused ports\\
 \hline 
 An adversary constructs a message that escapes HTML and runs code when rendered (XSS) & 3 & 5 & 15 & Sanitize user input\\
 \hline
  An adversary finds our Docker images online & 3 & 1 & 3 & Make image repository private\\ 
 \hline
 An adversary hijacks the cookie of a user & 1 & 2 & 2 & Unavoidable\\ 
 \hline
\end{tabular}
\end{center}

\noindent Unfortunately, we have not had time to implement the above actions. However, we have during the course made some hardening of our system e.g. made a secure connection to the DB and used GitHub secrets to store enviroment variables.