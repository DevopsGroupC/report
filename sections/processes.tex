\section{Processes}

% \subsubsection{Docker Swarm - Csharp-Minitwit application}
% When deploying a Docker Swarm cluster, a reserved IP is assigned to the manager node. Users will use this IP to send requests to the Csharp-Minitwit application. When receiving a request, the Docker Swarm cluster routes it through its routing mesh to one of the worker nodes running an instance of the Csharp-Minitwit application, which will process the request and serve a response.

\subsection{Tools and stages included in CI/CD pipelines}
\label{tools-and-stages-ci-cd}
% A complete description of stages and tools included in the CI/CD chains.
% That is, including deployment and release of your systems.
Our CI/CD pipelines are implemented using Github Actions. Our workflows are splitted in the following:

\subsubsection{quality.yml}
Triggered on pull request actions (opened, reopened, edited, synchronize, and ready for review). The main purpose is to run quality checks over the changes existing on the pull request branch. Figure \ref{fig:quality.yml} illustrates the stages of the quality workflow.

\begin{figure}[H]
    \centering
    \includegraphics[width=\textwidth]{images/figures/Quality.pdf}
    \caption{Flowchart illustrating the stages of the quality workflow}
    \label{fig:quality.yml}
\end{figure}

\noindent  Tools used in quality.yml:

\begin{itemize}
    \item .NET SDK (for running dotnet commands)
    \item dotnet format (for automatic formatting)
    \item Hadolint (for Dockerfile linting)
    \item SonarQube (for code quality and security analysis)
\end{itemize}

\subsubsection{release.yml}
Triggered on pushes to \texttt{main} branch. It uses a base \texttt{release.config.js} file at the root of the repository and it is meant to automate the generation of releases and its changelogs through the use of conventional commit messages. Figure \ref{fig:release.yml} illustrates the stages of the release workflow.

\begin{figure}[H]
    \centering
    \includegraphics[width=\textwidth]{images/figures/Release.pdf}
    \caption{Flowchart illustrating the stages of the release workflow}
    \label{fig:release.yml}
\end{figure}

\noindent  Tools used in release.yml:
\begin{itemize}
    \item Node.js (for running npm and semantic-release)
    \item Semantic Release (for versioning and changelog management)
\end{itemize}

\subsubsection{deploy.yml}
Triggered on pushes to \texttt{develop} and \texttt{main}. It internally checks the targeted environment \texttt{(staging} or \texttt{prod)} retrieving the corresponding secrets,pushing the application image to Dockerhub and deploying to Digital Ocean. Figure \ref{fig:deploy.yml} illustrates the stages of the deploy workflow.

\begin{figure}[H]
    \centering
    \includegraphics[width=\textwidth]{images/figures/Deploy.pdf}
    \caption{Flowchart illustrating the stages of the deploy workflow}
    \label{fig:deploy.yml}
\end{figure}

\noindent  Tools used in deploy.yml:
\begin{itemize}
    \item Docker (login, build, push actions)
    \item SSH (for deploying to servers)
\end{itemize}

% \noindent When developers work on the source code, and merge the new features into the main branch, it triggers a deployment workflow that updates the application's Docker images on Dockerhub. A later step uploads the Terraform state files for infrastructure management, which pulls the updated images and initiates the infrastructure update process, deploying changes to Digital Ocean.

% \subsection{Organization of the repository}
% % Organization of your repositor(ies).
% % That is, either the structure of of mono-repository or organization of artifacts across repositories.
% % In essence, it has to be be clear what is stored where and why.
% The repository is organised in two main folders: csharp-minitwit and infrastructure. The first contains all application specific logic, while the second holds all infrastructure related configuration files.

% By organising the repository in this manner, there is a clear separation of concerns, making it easier to manage the project. Developers concerning with the development of the application do not need to understand the intricacies of the infrastructure and vice versa. 

% \subsubsection{csharp-minitwit}
% Main folders and files explained:
% \begin{itemize}
%     \item Controllers: APIController and HomeController handle all the logic for each endpoint.
%     \item Databases: contains the \texttt{schema.sql} and \texttt{minitwit.db} files used to initialise the database in the development environment.
%     \item Metrics: contain all metrics logic and configuration.
%     \item Middlewares: currently only a CatchAllMiddleware required to catch incoming requests and record metrics.
%     \item Models: these represent the way the data is shaped in the application. %I do not completely understand the division into api and DTOs. We should probably explain why.
%     \item Services: include the business logic and service layer components. ORM logic is reflected under Repositories.
%     \item Views: templates representing the user interface.
% \end{itemize}

% \subsubsection{infrastructure}
% Main folders and files explained:
% \begin{itemize}
%     \item archive
%     \item grafana
%     \item scripts
%     \item ssh\_key
%     \item stack
% \end{itemize}
    
    
% \subsection{Applied branching strategy}
% We have decided to use Gitflow branching model\cite{nvie2010git}. In this model the repository holds two main branches:
% \begin{itemize}
%     \item \texttt{main}: represents the main branch where the source code of HEAD is always production-ready.
%     \item \texttt{develop}: it serves as an integration branch where source code of HEAD always reflects a state with the latest delivered development changes for the next release.
% \end{itemize}
% Next to these main branches there are some other supporting branches, being \texttt{feature} the most widely used. feature branches are meant to be used to develop new features for future releases. They usually branch off and back to develop.

% \subsection{Applied development process and tools supporting it}
% % For example, how did you use issues, Kanban boards, etc. to organize open tasks
% To organise the development process and make work visible we decided to capitalise Github's built in Project tab, which provides an adaptable spreadsheet for tracking work, and which also integrates with Issues and Pull Requests. This tool provides a comprehensive visibility to the current state of each ticket and an easy-to-understand interface for developers and stakeholders.

% A new Issue is created to document a new requirement and is automatically shown in the Project board as a "Todo" item. When a team member is assigned to such issue, it is moved into the "In Progress" state. The development phase takes part, and when the issue is closed, it is automatically moved to the "Done" status in the board. Closing an issue as "not planned" takes the ticket to the "Aborted" status.

% \begin{figure}[H]
%     \centering
%     \includegraphics[width=\textwidth]{images/github-project-tracking.png}
%     \caption{Github's Project tracking tool}
%     \label{fig:github-project-feature}
% \end{figure}

\subsection{Monitoring}
% How do you monitor your systems and what precisely do you monitor?
Monitoring is conducted using a self-hosted instance of Grafana on a DigitalOcean Droplet. This instance receives metrics data from Prometheus servers attached to the application droplet. When the Csharp-Minitwit application processes a request that involves a metric method, the OpenTelemetry.Exporter.Prometheus.AspNetCore package collects and exposes these metrics on a \textit{/metrics} endpoint. A Prometheus server scrapes this endpoint at regular intervals. Prometheus is then used as a data source for Grafana, allowing us to visualize this information on a dashboard.\newline

\noindent Some of the data is also collected by querying the database on some specific indicators, such as:
\begin{itemize}
    \item Messages registered (application usage)
    \item Users registered (conversion)
    \item Follower registrations (users interaction level)
\end{itemize}

\subsubsection{Application monitoring}
The data is then used together with the build in packages to display, how the application is performing. We have chosen to monitor the following, as seen in Figure \ref{fig:grafana-dashboard}.
\begin{itemize}
    \item Rate of HTTP requests received per endpoint
    \item Total number of requests (last 24hs)
    \item Total count of errors per status code (last 24hs).
    \item Top 10 unhandled exception endpoints.
    \item Top 10 Requested endpoints (API).
\end{itemize}

\begin{figure}[H]
    \centering
    \includegraphics[height=0.9\textwidth]{images/Monitoring-latest.png}
    \caption{Snapshot of Grafana Monitoring Dashboard}
    \label{fig:grafana-dashboard}
\end{figure}

\subsubsection{Infrastructure monitoring}
Even though we haven't set this monitoring ourselves, Digital Ocean provides some out of the box monitoring for its droplets, such as CPU usage, memory usage, DISK I/O, Disk Usage, Bandwith, etc.

\subsection{Logging}
% What do you log in your systems and how do you aggregate logs?
Logging is configured through a log aggregation system called Loki, which acts as a Grafana datasource. These logs are pushed directly to Loki by using Serilog, a logging library for .NET applications. Serilog configures a C\# API that will receive the Csharp-Minitwit application logs and send them in a POST request to Loki's server. The structured representation of these Logs makes it easier to extract information from them at a later point. Loki can then be configured as a data source in Grafana to process and store these logs to be queried from a Dashboard.

\noindent Below, we present the log events throughout our application:
\begin{itemize}
    \item Message posting
    \item User logging / Failure logging / Invalid username or password
    \item User logout
    \item User registering / Failure registering
    \item API requests that are not from the simulator
    \item API registering / Failure registering
    \item API message posting from users / Failure or incorrect usage of endpoint
    \item API retrieval of messages for a specific user / Failure in the retrieval
    \item API retrieval of followers for a specific user / Failure in the retrieval
    \item API follow/unfollow requests / Failure in the execution
    
\end{itemize}

% Concerning log aggregation, Loki does it in a different way compared to traditional logging systems. To be more precise, Loki adopts a more streamlined approach which entails indexing the labels and metadata from each message. Afterwards, the log message itself is compressed and available for querying in Grafana based on specific labels, time ranges and patterns. In the case of our group, we store it in chunks in a DigitalOcean volume attached to the Grafana droplet associated with our Csharp-Minitwit application.

\noindent Loki uses a streamlined approach for log aggregation by indexing labels and metadata from each message, then compressing the log message itself. This allows for efficient querying in Grafana based on labels, time ranges, and patterns. For our Csharp-Minitwit application, logs are stored in chunks on a DigitalOcean volume attached to the Grafana droplet.

\subsection{Security assessment}

Having multiple assets, we have multiple threat sources.\newline 
We have the following assets:
\begin{itemize}
    \item Web application
    \item Database
    \item CI/CD
    \item Image repository
    \item Container orchestration
    \item Secrets
\end{itemize}

\subsubsection{Threat sources}

Web application: Runs on HTTP, making it vulnerable to data leaks and replay attacks. Possible XSS vulnerabilities due to HTML rendering.\\\\
Database: Usernames and emails are stored in clear-text; passwords are salted and hashed, to defend against rainbow-table attacks.
User IDs increment sequentially, which is not best practice.The same database credentials are used for both staging and production.\\\\
CI/CD: GitHub actions has access to all secrets.
We use the following NPM packages for release versioning:
\begin{itemize}
    \item "semantic-release-github-actions-tags": "\textasciicircum1.0.3"
    \item "semantic-release/git": "\textasciicircum8.0.0"
\end{itemize}

Images are publicly stored on DockerHub, with clear-text credentials on the staging server. The codebase is public on GitHub.\\\\
Other: Multiple open ports on servers, potentially including logging endpoints.

\subsubsection{Risk scenarios}

\begin{center}
\begin{tabular}{ |p{3.5cm}|c|c|c|p{2.5cm}| } 
 \hline
 Scenario & Probability & Impact & Risk & Strategy / Action\\ [0.5ex] 
 \hline
 An adversary listens to the non-encrypted data sent on our web application. This is possible because we use HTTP and not HTTPS
 & 5 & 5 & 25 & Get an SSL-certificate and redirect all traffic to HTTPS\\
\hline
 An adversary finds an open port on one of our nodes and gains access to our system & 3 & 5 & 15 & Run Nmap, close unused ports\\
 \hline 
 An adversary constructs a message that escapes HTML and runs code when rendered (XSS) & 3 & 5 & 15 & Sanitize user input\\
 \hline
  An adversary finds our Docker images online & 3 & 1 & 3 & Make image repository private\\ 
 \hline
 An adversary hijacks the cookie of a user & 1 & 2 & 2 & Unavoidable\\ 
 \hline
\end{tabular}
\end{center}


\noindent We have not had time to harden our system, due to our focus primarily being on making the infrastructure. 

\subsection{Scaling and upgrades}
We monitor our system to determine when scaling is needed. To scale, we adjust the number of nodes in the Terraform file `minitwit\textunderscore swarm\textunderscore cluster.tf` and run the `bootstrap.sh` script. This process is quick due to the state being stored in DigitalOcean Spaces Object Storage.

Currently, we lack an automated upgrade process. Based on our experience upgrading from Python 2 to Python 3, we know upgrades are complex and require careful management of dependencies.

\subsection{AI-assistance}
AI tools have been part of our development process, with ChatGPT mainly used for debugging and advice. However, due to the unique nature of our application, ChatGPT's resources were limited. We found the documentation for each dependency more thorough and effective for addressing our specific implementation needs.
